\documentclass[macos]{exam-cau}

\newcommand{\mb}{\bb}
\newcommand{\mc}{\mathcal}   
\newcommand{\Lim}{\lim\limits}
\newcommand{\Liminf}{\liminf\limits}

% 控制是否显示答案
\includeanswertrue   % 显示答案
% \includeanswerfalse  % 不显示答案

\begin{document}

% 考试信息设置
\setyear{2023}
\setsubject{模拟课程}
\setsemester{春季学期}

% 生成试卷标题等 
\generateExamTitle

% 一、填空题 
\problem{24}{填空题}{
     \begin{enumerate}
    \item [1.] $\Lim_{x\to 0}(\cos x)^{1/x^2}=$\underline{\quad\quad\quad\quad}
      \vspace{0.2in}
      \begin{answer}
        \begin{align*}
          \Lim_{x\to 0}(\cos x)^{1/x^2} = 
          \score{3}
        \end{align*}
        
        \score{3}
      \end{answer}
    \item [2.] 当 $p>0$ 时, $x^3+px+q=0$ 有\underline{\quad\quad\quad\quad}个实根
      \vspace{0.2in}
      
    \item [3.] 当 $x\to+\infty$ 时, 试将下述无穷大量按由低阶至高阶的顺序排列:\par
      $e^x, \, x^x,\, x^{100},\,x^{99}(\ln x)^{100},\,
      [x]!$ \underline{\quad\quad\quad\quad\quad\quad\quad\quad\quad\quad\quad\quad}
      \vspace{0.2in}

    \item [4.] $\int_0^\pi \cos^2 x dx =$ \underline{\quad\quad\quad\quad}
      \vspace{0.2in}

    \item [5.]
      $\left.\frac{d}{dx}\right|_{x=1}\frac{\sqrt{x}}{1+2x}=$\underline{\quad\quad\quad\quad}
      \vspace{0.2in}

    \item [6.] 求 $\Liminf_{n\to\infty}D(\frac1{\sqrt
        {n+1}})=$\underline{\quad\quad\quad\quad}, 其中 $D(x)$ 为 Dirichlet 函数,
      即
      $$D(x)=
      \begin{cases}
        1 & x\in \mathbb{Q}\\
        0 & x\not\in \mathbb{Q}
      \end{cases}.
      $$
      \vspace{0.2in}

    \item [7.] 求 $\frac{d^n}{dx^n}(x^2e^x)=$\underline{\quad\quad\quad\quad\quad\quad\quad\quad},
      ($n\in\mathbb{N}$,化简所得结果)
      \vspace{0.2in}
 
    \item [8.] 下列关于一致连续的说法中,正确的有多少个?\underline{\quad\quad\quad\quad}
      \begin{enumerate}
      \item 若$f(x)$在$(a,b)$连续,则对充分小的$\delta>0$,$f(x)$在
        $[a+\delta,b-\delta]$上一致连续
      \item 若$f(x)$在$(a,b)$连续,则在$(a,b)$上有界
      \item 若$f(x)$在$(a,b)$上一致连续,则在$(a,b)$上有界
      \item $\ln(x)$在$(1,+\infty)$上一致连续
      \item 某区间上两个一致连续的函数之和一定一致连续
      \end{enumerate}
      (注: $a,b$ 均为有限值)
    \end{enumerate}

} % 第一大题结束 

% 二、计算题
\problem{24}{计算题}{
    \begin{enumerate}
    \item [1.] $$\int \cos^2(x)\sin(x)dx$$ \vspace{2.5in}
    \item [2.] $$\int \frac x{\sqrt{1-x^2}}dx$$ \vspace{2.5in}
      \newpage
    \item [3.] $$\int \frac {-x^4+x^3-x^2-x-2}{(x^2+1)^2(x-1)}dx$$ \vspace{4in}
    \item [4.] $$\int \sin(\ln x)dx$$\vspace{2in}
    \end{enumerate}
    \newpage
} % 第二大题结束 

% 第三大题  
\problem{6}{}{
  求 $a,b$, 使
    $$f(x)=
    \begin{cases}
      ax+b & x>1\\
      x^2-3x+2 & x\le 1
    \end{cases}
    $$
    为可微函数.
    \vspace{2.5in}
} % 第三大题结束 

% 第四大题 
\problem{6}{}{ 
  对于$\mathbb{R}$上有定义的函数, 若所论的导函数存在, 证明结论:\\
    奇函数的导函数一定是偶函数.  \vspace{2in} \newpage
} % 第四大题结束 

% 第五大题 
\problem{10}{}{
求过曲线 $$x^{2n}+y^{2n}=1$$ 上 $(x_0,y_0)$ 点的切线方
    程(其中 $n$ 为自然数, $y_0\neq0$). 并证明当 $n\to+\infty$ 时, 除有限个点外,
    $y'(x)$ 要么趋于 $0$, 要么趋于 $\infty$. (注: 实际上随着 $n$ 的增加, 曲线越
    来越接近于正方形)  \vspace{3in}
} % 第五大题结束 

% 第六大题 
\problem{10}{}{
设 $a<b$, $f(x)$ 在 $(-\infty,b)$ 和 $(a,+\infty)$ 均
    一致连续, 证明 $f(x)$ 在 $(-\infty,+\infty)$ 上也一致连续.
    \vspace{2in} \newpage
} % 第六大题结束 

% 第七大题 
\problem{10}{}{
设 $f(x)$ 在 $\mathbb{R}$ 上连续, $f(1)>0$, 且
    $\Lim_{x\to\pm\infty}f(x)=0$, 证明 $f(x)$ 在 $\mathbb{R}$ 上有最大值.
    \vspace{3in}
} % 第七大题结束 

% 第八大题 
\problem{10}{}{
用 Bolzano-Weierstrass 定理证明有界闭区间上的连续函数一定有
    界.\vspace{2in}
} % 第八大题结束 

% 输出表格和正文
\generateProblemTable

\end{document}



